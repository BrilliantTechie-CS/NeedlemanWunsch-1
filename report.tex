%        File: report.tex
%     Created: Tue Apr 10 01:00 PM 2012 E
% Last Change: Tue Apr 10 01:00 PM 2012 E
%
\documentclass[a4paper]{report}
\begin{document}

% the '*'s suppress auto-numbering of sections
\section*{Code Details}
\subsection*{DNA Object}

The DNA object is relatively simple. Its only responsibility is to validate
and store a sequence of characters representing a piece of DNA. Inputting
data into the DNA object is done either when the object is constructed, by
passing a c-string representing the sequence to the constructor, or
using an overloaded \verb=operator>>= function reading from \verb=cin= (or any other \verb=istream=
 object).

Validation of the data is done as soon as it is received. The validation
process is simple - each character is checked, and if any is not a member
of the set $\{\verb=A=,\verb=C=,\verb=G=,\verb=T=\}$ then the sequence is
invalid, and an exception is thrown that contains information about where in the
sequence the invalid character appeared.

Data from the DNA object is readable through the object's \verb=operator[]= function.

\subsection*{DPM Object}

\subsubsection*{Accomodating Different Data Types}

The DPM object is considerably more complex than the DNA object. Because the
data type that is used internally by the DPM object does not have any affect
on the public interface for getting the results of the algorithm, the
implementation details of the templated class were hidden as much as possible
from the end user of the DPM object. This was done by wrapping all of
the business logic of the Needleman-Wunsch algorithm in a separate template class,
\verb=DPM_Implementation=, that inherits from an abstract base class \verb=DPM_ImplementationBase=.
The actual DPM object only keeps track of a \\* \verb=DPM_ImplementationBase= pointer
that references an instance of the templated \verb=DPM_Implementation= class. By
using one constructor that takes \verb=int= parameters, and another that takes \verb=double=s,
the user of the class may pass whatever datatype they like, and not have to deal
with variables of multiple templated types.

An additional advantage of this approach is that it makes it possible to compile
the templated implementation of the Needleman-Wunsch algorithm separately from the
code that actually uses it, removing the necessity to \verb=#include= the
implementation file, and decreasing the physical dependency between the DPM class and
programs that use it.

\subsubsection*{Traceback Procedure}

To optimize the traceback procedure, data about possible paths through
the matrix is stored during the filling of the matrix. Each cell of the matrix
is represented by a simple \verb=struct= that contains the score for that position
in the matrix, and a set of flags indicating the cells that the current cell's
score may have originated from. This does not increase the space or time
complexity of the filling stage. The amount of space used increases from
$4mn$ bytes to $5mn$ bytes, where $m$ and $n$ are the lengths of the
two sequences. Similarly, calculating the flags for each cell only adds a
constant amount of time to the calculations done for each cell, and the actual
time complexity does not change.

Since the direction information is already stored within the matrix, the
traceback procedure becomes a simple traversal of the acyclic directed
graph made up of the flags stored in the matrix. This traversal could easily
be done using a recursive function, but instead a stack-based implementation
was chosen. The reason for this was to make it possible to retrieve only a
limited number of possible alignments, or to retrieve all alignments using
a lazy evaluation strategy, which can provide significant memory savings.

To make the retrieval of alignments as convenient as possible, an STL compatible
iterator was created that makes it possible to easily retrieve results one at a
time. The iterator keeps a stack of nodes, that act as a bookmark for the current
position within the traceback procedure. Each time a new alignment is requested,
the iterator calls the \verb=_traceback()= function of its parent DPM object, passing its
current stack. The stack is updated, and the next alignment is returned to the
iterator, overwriting the alignment that it stored previously.

In this program, the main advantage of this approach to calculating all
possible alignments is reduced space complexity. Rather than calculating all
alignments, storing them, and then looping over all of them, only one alignment
object is ever created at a time, allowing them to be looped over in constant
space relative to the number of optimal alignments. This is useful, because often
there may be tens or hundreds of thousands of possible alignments.

\subsection*{Time and Space Complexity}

\subsubsection{Matrix Initialization and Filling}

The time and space complexity of the matrix filling and initialization are both
quadratic. A constant amount of time and space is needed to calculate the scores and
directional flags at each cell in the matrix, and there are $(m + 1)(n + 1)$ cells
in the matrix in total, giving time and space complexities of $O(mn)$.

\subsubsection*{Traceback Procedure}

The time complexity of the traceback procedure is different for retrieving
a single optimal alignment versus all possible alignments. For a single optimal
alignment, the maximum possible length is $m + n$, which would result from a situation
where there are no matches and gaps are inserted at all positions, for example:

\begin{verbatim}
AAAAAAAAA-------
---------GGGGGGG
\end{verbatim}

In this case, the time complexity is $O(m + n)$. The best case is that the length of
the optimal alignment is the same as the length of the longest sequence to be aligned,
for example:

\begin{verbatim}
ATCGTAGCTCAGTACGTGACTG
AT--TAGC-CAG-----GA-TG
\end{verbatim}

In this case, the time complexity is $O(max(m,n))$. In both cases, the time complexity is
roughly linear in $m$ and $n$.

The time complexity for finding all optimal alignments is obviously greater than for
finding one optimal alignment. In this case the directed acyclic graph which is implicitly
stored in the filled matrix (represented by the directional flags at each cell) is used,
which makes finding all optimal alignments a simple depth-first traversal of the graph.
The worst-case time complexity is therefore $O(E + V)$ where $E$ is the number of edges
in the graph and $V$ is the number of vertices. In the worst case, every cell in the matrix
is a vertex, and every cell (other than those on the sides of the matrix) has three edges
leaving it (with the $m + n$ edge cells having one edge each). This gives a total worst-case
time complexity of $O((m + 1)(n + 1) + 3mn + m + n) = O(mn)$.

The space complexity of finding a single optimal alignment, or of finding all optimal
alignments is the same -- $O(m + n)$. This is the case for the same reason that the time
complexity of finding a single optimal alignment is $O(m + n)$. For each character in the
optimal alignment, one character must be appended to each of the strings in the optimal alignment,
and between one and three cells must be pushed onto the stack. This gives a total maximum
space usage of $(m + n)(2 \verb=sizeof(char)= + 3 \verb=sizeof(stackCell)=)$, which corresponds to a space
complexity of $O(m + n)$. The space complexity does not increase when all optimal alignments are
retrieved, because only one alignment is stored at a time.



\subsection*{Bugs}

There are no known bugs in the program. It compiles without warnings using
the \verb=g++= flags \verb=-Wall= and \verb=-Wextra=, and
testing with \verb=valgrind= has shown that there are no memory leaks.

\section*{Observations}

For the most part, variation of the match score, gap penalty and mismatch penalty provide
predictable results. The higher the match score is set, the more matches are valued in alignments, at
the expense of adding more gaps or mismatches. Consider the following alignment of the sequences
\verb=GAGAGTTT= and \verb=CCCGAGAG=. With a match score of 1, a gap penalty of 1 and a mismatch penalty of
0, the first optimal alignment is:

\begin{verbatim}
-GAGAGTTT
CCCGAG-AG
\end{verbatim}

In the above alignment, not all possible matches are made, because the cost of the addition of four gaps (-4) is
greater than the reward from adding two more matches (+2). If the match score is increased to 2, then both of the
following alignments are considered equally optimal:

\begin{verbatim}
-GAGAGTTT
CCCGAG-AG

---GAGAGTTT
CCCGAGAG---
\end{verbatim}

Finally, by increasing the match score to 3, the second alignment, which contains more matches, becomes better
than the first. In general, by making the match score sufficiently high, the optimal alignments will always
contain the maximum possible number of matches.

Similarly, by making the mismatch score sufficiently high, and the gap penalty sufficiently low,
it is possible to obtain optimal alignments that contain all possible matches, and no mismatches.
This type of alignment is actually a solution to the longest-common-subsequence problem. For example,
in the following alignment, the longest common subsequence between the two sequences \verb=GCCTACAATCGTACGCA=
and \verb=CTATCTACCTGTACTTG= is found by simply ignoring all gaps in the alignment.

\begin{verbatim}
GC---CTA-CAATCGTAC--GCA
-CTATCTACC--T-GTACTTG--
 C   CTA C  T GTAC  G
\end{verbatim}

Which gives a longest-common-subsequence of \verb=CCTACTGTACG=. There are also other
possible longest-common-subsequences, such as \verb=CTACACGTACG= which is obtained
from the alignment:

\begin{verbatim}
GCCTA-CA-AT-C-GTAC--GCA
-C-TATC-TA-CCTGTACTTG--
 C TA C  A  C GTAC  G
\end{verbatim}

Additionally, optimal alignments producing longest-common-subsequences are not necessarily unique.
For example, the following alignment also gives the longest-common-subsequence \verb=CCTACTGTACG=.

\begin{verbatim}
GC---CTAC-AATCGTAC--GCA
-CTATCTACC--T-GTACTTG--
 C   CTAC   T GTAC  G
 \end{verbatim}

\end{document}


